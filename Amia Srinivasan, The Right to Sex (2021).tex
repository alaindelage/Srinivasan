% \documentclass[a4paper]{report}
\input{preamble}
\input{hyphenation}

%\setlength{\baselineskip}{2ex}
% \renewcommand{\baselinestretch}{1.2}
\selectlanguage{ngerman}
\usepackage{fourier}

\begin{document}
% \doublespacing
% \onehalfspacing
\thispagestyle{empty}

% einzeiliger Titel

% {\Large AUTOR\ \ \emph{TITEL} (JAHR)}

% oder mehrzeiliger Titel

\begin{tabbing}
	{\Large Amia Srinivasan} \ \ \={\Large \emph{The Right to Sex}}\\[1ex]
	\> {\large \emph{Feminism in the Twenty-First Century} (Bloomsbury 2021)}
\end{tabbing}
\vskip 5ex


% \flushleft
\RaggedRight

% \changefont{ptm}{m}{n}
\begin{center}
	{\large Preface}
\end{center}
\vskip 3ex

feminism: a political movement, not a theory
\begin{quote}
	\textbf{Feminism} is \textbf{not a philosophy}, or \textbf{a theory}, or even \textbf{a point of view}.\\[1ex] 
	It is \textbf{a political movement to transform the world beyond recognition}. It asks: what would it be to end the political, social, sexual, economic, psychological and physical subordination of women? \textbf{It answers: we do not know; let us try and see}. xi
\end{quote}
\vskip 2ex

‘sex’
\begin{quote}
	Feminism begins with \textbf{a woman’s recognition that she is a member of a sex class}: that is, a member of a class of people \textbf{assigned to an inferior social status on the basis of something called ‘sex’} -- \textbf{a thing that is said to be natural, pre-political, an objective material ground on which the world of human culture is built}. xi
\end{quote}
\vskip 2ex

‘sex’, this supposedly natural thing -- a cultural thing posing as a natural one
\begin{quote}
	We inspect \textbf{this supposedly natural thing, ‘sex’}, only to find that it is \textbf{already laden with meaning}.\\[1ex] 
	At birth, bodies are sorted as ‘male’ or ‘female’, though many bodies must be mutilated to fit one category or the other, and many bodies will later protest against the decision that was made.\\[1ex] 
	\textbf{This originary division determines what social purpose a body will be assigned}. [\dots\hspace{-0.3ex}]\\[1ex]
	\textbf{Sex} is, then, \textbf{a cultural thing posing as a natural one}. Sex, which feminists have taught us to distinguish from gender, is \textbf{itself already gender in disguise}. xi f.
\end{quote}
\vskip 2ex

‘sex’, in another sense: a thing we do with our sexed bodies
\begin{quote}
	\textbf{‘sex’}: sex as \textbf{a thing we do with our sexed bodies}. Some bodies are for other bodies to have sex with. [\dots\hspace{-0.3ex}]\\[1ex] 
	‘Sex’ in this second sense is also \textbf{said to be a natural thing, a thing that exists outside politics}. Feminism shows that \textbf{this too is a fiction}, and a fiction that serves certain interests. Sex, which we think of as the most private of acts, is \textbf{in reality a public thing}. [\dots\hspace{-0.3ex}]\\[1ex]
	the rules for all this were set long before we entered the world xii
\end{quote}
\vskip 2ex

feminism ans sexual freedom
\begin{quote}
	Feminists have long dreamed of \textbf{sexual freedom}. What they refuse to accept is \textbf{its simulacrum: sex that is said to be free, not because it is equal, but because it is ubiquitous}. In this world, sexual freedom is \textbf{not a given but something to be achieved}, and it is \textbf{always incomplete}. xii
\end{quote}
\vskip 2ex

\begin{quote}
	\textbf{What would it take for sex really to be free?} We do not yet know; let us try and see. xiii
\end{quote}
\vskip 3ex

sex as a political phenomenon -- beyond the narrow parameters of ‘consent’
\begin{quote}
	These essays are \textbf{about the politics and ethics of sex in this world}, animated by a hope of a different world.\\[1ex] 
	They reach back to \textbf{an older feminist tradition that was unafraid to think of sex as a political phenomenon}, as something \textbf{squarely within the bounds of social critique}. The women in this tradition -- from Simone de Beauvoir and Alexandra Kollontai to bell hooks, Audre Lorde, Catharine MacKinnon and Adrienne Rich – dare us to think about the ethics of sex \textbf{beyond the narrow parameters of ‘consent’}. They compel us to ask \textbf{what forces lie behind a woman’s \emph{yes}}; what it reveals about sex that it is something to which consent must be given; how it is that we have come to put so much psychic, cultural and legal weight on a notion of ‘consent’ that cannot support it.\\[1ex] 
	And they ask us to join them in \textbf{dreaming of a freer sex}. xiii
\end{quote}
\vskip 2ex

remake the political c tique of sex for the twenty-first century
\begin{quote}
	At the same time, these essays seek to \textbf{remake the political critique of sex for the twenty-first century}: to take seriously \textbf{the complex relationship of sex to race, class, disability, nationality and caste}; to think about what sex has become \textbf{in the age of the internet}; to ask what it means to \textbf{invoke the power of the capitalist and carceral state} to address the problems of sex. xiii f.
\end{quote}
\vskip 2ex

feminism not as ‘home’
\begin{quote}
	Feminism cannot indulge the fantasy that interests always converge; [\dots\hspace{-0.3ex}]\\[1ex]
	\textbf{Feminism envisaged as a ‘home’ insists on commonality before the fact} [\dots\hspace{-0.3ex}] A truly inclusionary politics is an uncomfortable, unsafe politics. xv
\end{quote}
\vskip 2ex

\begin{quote}
	they [the essays] represent my attempt to put into words what many women, and some men, already know xv
\end{quote}
\vskip 5ex













\end{document}