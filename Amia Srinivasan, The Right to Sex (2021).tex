% \documentclass[a4paper]{report}
\input{preamble}
\input{hyphenation}

%\setlength{\baselineskip}{2ex}
% \renewcommand{\baselinestretch}{1.2}
\selectlanguage{ngerman}
\usepackage{fourier}

\begin{document}
% \doublespacing
% \onehalfspacing
\thispagestyle{empty}

% einzeiliger Titel

% {\Large AUTOR\ \ \emph{TITEL} (JAHR)}

% oder mehrzeiliger Titel

\begin{tabbing}
	{\Large Amia Srinivasan} \ \ \={\Large \emph{The Right to Sex}}\\[1ex]
	\> {\large \emph{Feminism in the Twenty-First Century} (Bloomsbury 2021)}
\end{tabbing}
\vskip 5ex


% \flushleft
\RaggedRight

% \changefont{ptm}{m}{n}
\begin{center}
	{\large Preface}
\end{center}
\vskip 3ex

feminism: a political movement, not a theory
\begin{quote}
	\textbf{Feminism} is \textbf{not a philosophy}, or \textbf{a theory}, or even \textbf{a point of view}.\\[1ex] 
	It is \textbf{a political movement to transform the world beyond recognition}. It asks: what would it be to end the political, social, sexual, economic, psychological and physical subordination of women? \textbf{It answers: we do not know; let us try and see}. xi
\end{quote}
\vskip 2ex

‘sex’
\begin{quote}
	Feminism begins with \textbf{a woman’s recognition that she is a member of a sex class}: that is, a member of a class of people \textbf{assigned to an inferior social status on the basis of something called ‘sex’} -- \textbf{a thing that is said to be natural, pre-political, an objective material ground on which the world of human culture is built}. xi
\end{quote}
\vskip 2ex

‘sex’, this supposedly natural thing -- a cultural thing posing as a natural one
\begin{quote}
	We inspect \textbf{this supposedly natural thing, ‘sex’}, only to find that it is \textbf{already laden with meaning}.\\[1ex] 
	At birth, bodies are sorted as ‘male’ or ‘female’, though many bodies must be mutilated to fit one category or the other, and many bodies will later protest against the decision that was made.\\[1ex] 
	\textbf{This originary division determines what social purpose a body will be assigned}. [\dots\hspace{-0.3ex}]\\[1ex]
	\textbf{Sex} is, then, \textbf{a cultural thing posing as a natural one}. Sex, which feminists have taught us to distinguish from gender, is \textbf{itself already gender in disguise}. xi f.
\end{quote}
\vskip 2ex

‘sex’, in another sense: a thing we do with our sexed bodies
\begin{quote}
	\textbf{‘sex’}: sex as \textbf{a thing we do with our sexed bodies}. Some bodies are for other bodies to have sex with. [\dots\hspace{-0.3ex}]\\[1ex] 
	‘Sex’ in this second sense is also \textbf{said to be a natural thing, a thing that exists outside politics}. Feminism shows that \textbf{this too is a fiction}, and a fiction that serves certain interests. Sex, which we think of as the most private of acts, is \textbf{in reality a public thing}. [\dots\hspace{-0.3ex}]\\[1ex]
	the rules for all this were set long before we entered the world xii
\end{quote}
\vskip 2ex

feminism ans sexual freedom
\begin{quote}
	Feminists have long dreamed of \textbf{sexual freedom}. What they refuse to accept is \textbf{its simulacrum: sex that is said to be free, not because it is equal, but because it is ubiquitous}. In this world, sexual freedom is \textbf{not a given but something to be achieved}, and it is \textbf{always incomplete}. xii
\end{quote}
\vskip 2ex

\begin{quote}
	\textbf{What would it take for sex really to be free?} We do not yet know; let us try and see. xiii
\end{quote}
\vskip 3ex

sex as a political phenomenon -- beyond the narrow parameters of ‘consent’
\begin{quote}
	These essays are \textbf{about the politics and ethics of sex in this world}, animated by a hope of a different world.\\[1ex] 
	They reach back to \textbf{an older feminist tradition that was unafraid to think of sex as a political phenomenon}, as something \textbf{squarely within the bounds of social critique}. The women in this tradition -- from Simone de Beauvoir and Alexandra Kollontai to bell hooks, Audre Lorde, Catharine MacKinnon and Adrienne Rich – dare us to think about the ethics of sex \textbf{beyond the narrow parameters of ‘consent’}. They compel us to ask \textbf{what forces lie behind a woman’s \emph{yes}}; what it reveals about sex that it is something to which consent must be given; how it is that we have come to put so much psychic, cultural and legal weight on a notion of ‘consent’ that cannot support it.\\[1ex] 
	And they ask us to join them in \textbf{dreaming of a freer sex}. xiii
\end{quote}
\vskip 2ex

remake the political c tique of sex for the twenty-first century
\begin{quote}
	At the same time, these essays seek to \textbf{remake the political critique of sex for the twenty-first century}: to take seriously \textbf{the complex relationship of sex to race, class, disability, nationality and caste}; to think about what sex has become \textbf{in the age of the internet}; to ask what it means to \textbf{invoke the power of the capitalist and carceral state} to address the problems of sex. xiii f.
\end{quote}
\vskip 2ex

feminism not as ‘home’
\begin{quote}
	Feminism cannot indulge the fantasy that interests always converge; [\dots\hspace{-0.3ex}]\\[1ex]
	\textbf{Feminism envisaged as a ‘home’ insists on commonality before the fact} [\dots\hspace{-0.3ex}] A truly inclusionary politics is an uncomfortable, unsafe politics. xv
\end{quote}
\vskip 2ex

\begin{quote}
	they [the essays] represent my attempt to put into words what many women, and some men, already know xv
\end{quote}
\vskip 5ex
\pagebreak

\begin{center}
	{\large The Conspiracy Against Men}
\end{center}
\vskip 3ex

false rape accusation -- and its cultural charge
 \begin{quote}
 	Nonetheless, a false rape accusation, like a plane crash, is an objectively unusual event that occupies an outsized place in the public imagination.\\[1ex]
 	Why then does it carry its cultural charge? 3
 \end{quote}
 \vskip 2ex
 
 false rape accusation, by men
\begin{quote}
	very often, it is men who falsely accuse other men of raping women. This is a thing almost universally misunderstood about false rape accusations. When we think of a false rape accusation we picture a scorned or greedy woman, lying to the authorities. But many, perhaps most, wrongful convictions of rape result from false accusations levied against men by other men 4
\end{quote}
\vskip 2ex

false rape accusations as a predominantly wealthy white male preoccupation
\begin{quote}
	It might seem surprising, then, that false rape accusations are, today, a predominantly wealthy white male preoccupation.\\[1ex]
	But it isn’t surprising -- not really. The \textbf{anxiety about false rape accusations} is purportedly about injustice (innocent people being harmed), but \textbf{actually it is about gender, about innocent men being harmed by malignant women}. It is \textbf{an anxiety}, too, \textbf{about race and class}: about \textbf{the possibility that the law might treat wealthy white men as it routinely treats poor black and brown men}.\\1ex 
	For poor men, and women, of colour, the white woman’s false rape accusation is just one element in a matrix of vulnerability to state power.\\[1ex]
	false rape accusations are \textbf{a unique instance of middle-class and wealthy white men’s vulnerability} to the injustices routinely perpetrated by the carceral state against poor people of colour. 5f.
\end{quote}
\vskip 2ex

the representation is false -- but, as ideologically efficacious
\begin{quote}
	That representation is, of course, false: even in the case of rape, the state is on the side of wealthy white men.\\[1ex] 
	But what matters in the sense of what is ideologically efficacious -- is not the reality, but the misrepresentation. \textbf{In the false rape accusation, wealthy white men misperceive their vulnerability to women and to the state}. 6
\end{quote}
\vskip 2ex

Brock Turner case (Santa Clara County, 2016)
\begin{quote}
	‘20 minutes of action’ -- healthy, adolescent fun [\dots\hspace{-0.3ex}]\\[1ex]quotein a sense Dan Turner is talking about an animal, a p fectly bred specimen of wealthy white American boyhood [\dots\hspace{-0.3ex}]\\[1ex]
	like an animal, Brock is imagined to exist outside the moral order.These red-blooded, white-skinned, all-American boys [\dots\hspace{-0.3ex}] are good kids, the best kids, our kids. 7
\end{quote}
\vskip 2ex

Brett Kavanaugh
\begin{quote}
	The solidarity on show from the people who knew Kavanaugh when young – what Kavanaugh calls ‘friendship’ -- was the solidarity of rich white people.\\[1ex] 
	We can’t imagine a black or brown Kavanaugh without inverting America’s racial and economic rules. 9
\end{quote}
\vskip 2ex

‘Believe women’, \#IBelieveHer
\begin{quote}
	Whom are we to believe, the white woman who says she was raped, or the black or brown woman who insists that her son is being set up? Carolyn Bryant or Mamie Till? 9
\end{quote}
\vskip 2ex

dismissal of ‘Believe women’ as a category error
\begin{quote}
	a political response to what we suspect will be its [legal principle of the presumption of innocence] uneven application [\dots\hspace{-0.3ex}]\\[1ex]
	Against this prejudicial enforcement of the presumption of innocence, ‘Believe women’ operates as a corrective norm, a gesture of support for those people – women – whom the law tends to treat as if they were lying. 9
\end{quote}
\vskip 2ex

a category error in a second sense
\begin{quote}
	The law must address each individual on a case-by-case basis [\dots\hspace{-0.3ex}] but the norms of the law do not set the norms of rational belief. Rational belief is proportionate to the evidence [\dots\hspace{-0.3ex}]\\[1ex]
	the outcome of a trial does not determine what we should believe 10
\end{quote}
\vskip 2ex

why sex crimes elicit such selective scepticism?
\begin{quote}
	The question, from a feminist perspective, is \textbf{why sex crimes elicit such selective scepticism}.\\[1ex] 
	And the answer that feminists should give is that \textbf{the vast majority of sex crimes are perpetrated by men against women}.\\[1ex] 
	Sometimes, the injunction to ‘Believe women’ is simply the injunction to form our beliefs in the ordinary way: in accordance with the facts. 10f.
\end{quote}
\vskip 2ex

\begin{quote}
	Does ‘Believe women’ serve justice at Colgate [University, elite liberal arts college; 4.2 per cent of the student body black in the academic year 2013-14; yet 50 per cent of accusations of sexual violation against black students]? 11
\end{quote}
\vskip 2ex

Jyoti Singh, 16 December 2012, Delhi
\begin{quote}
	the brutality of the attack on Jyoti Singh was cited by non-Indians as a way of disavowing any commonality between the sexual cultures of India and their own countries. [\dots\hspace{-0.3ex}]\\[1ex]
	A first question: why is it that when white men rape they are violating a norm, but when brown men rape they are conforming to one?\\[1ex]
	A second question: if Indian men are hyenas, what does that make Indian women? 12
\end{quote}
\vskip 2ex

the spectacle of the black male corpse, v/ the lack of the spectacle of the black female corpse
\begin{quote}
	‘What,’ Threadcraft asks, ‘will motivate people to rally around the bodies of our black female dead?’ 14
\end{quote}
\vskip 2ex

the white mythology about black sexuality
\begin{quote}
	portraying black men as rapists and black women as unrapeable 14
\end{quote}
\vskip 2ex

the doubled sexual subordination of black women
\begin{itemize}
	\item Black women who speak out against black male violence are blamed for reinforcing negative stereotypes of their community and for calling on a racist state to protect them. 

	\item At the same time, the internalisation of the sexually precocious black girl stereotype means that black girls and women are seen by some black men as asking for their abuse. 14
\end{itemize}
\vskip 2ex

Clarence Thomas v/ Anita Hill
\begin{quote}
	Fairfax did not note the irony of comparing black women to a white lynch mob.\\[1ex] Neither did Clarence Thomas, for that matter, when he accused Anita Hill in 1991 of triggering a ‘high-tech lynching’.\\[1ex] 
	\textbf{The very logic that made the lynching of black men possible -- the logic of black hypersexuality -- is repurposed, at the level of metaphor, to falsely indict black women as the true oppressors}. 15
\end{quote}
\vskip 2ex

intersectionality’s central insight: the danger of a lib movement’s serving the least suppressed members of a group
\begin{quote}
	\textbf{‘Intersectionality’} -- a term coined by \textbf{Kimberlé Crenshaw} to name an idea first articulated by an older generation of feminists from Claudia Jones to Frances M. Beal, the Combahee River Collective, Selma James, Angela Davis, bell hooks, Enriqueta Longeaux y Vásquez and Cherríe Moraga -- is \textbf{often reduced}, in common understanding,\textbf{ to a due consideration of the various axes of oppression and privilege: race, class, sexuality, disability and so on}. But to reduce intersectionality to a mere attention to difference is to forego its power as a theoretical and practical orientation.\\[1ex]
	\textbf{The central insight of intersectionality is that any liberation movement -- feminism, anti-racism, the labour movement -- that focuses only on what all members of the relevant group (women, people of colour, the working class) have in common is a movement that will best serve those members of the group who are least oppressed}. 17
\end{quote}
\vskip 2ex

the dilemma of ‘Believe women’
\begin{quote}
	When we are too quick to believe a white woman’s accusation against a black man, or a Brahmin woman’s accusation against a Dalit man, it is black and Dalit women who are rendered more vulnerable to sexual violence. Their ability to speak out against the violence they face from men of their race or caste is stifled, and their status as counterpart to the oversexed black or Dalit male is entrenched. In that paradox of female sexuality, such women are rendered unrapeable, and thus more rapeable. 18
\end{quote}
\vskip 2ex

\#Metoo, or, the rules changed suddenly
\begin{quote}
	This idea -- that the rules have suddenly changed on men, so that they now face punishment for behaviour that was once routinely permitted -- has become a \#MeToo commonplace. 20\\[1ex]
	\#MeToo is often seen as having produced a generalised version of the situation in which John Cogan found himself. Patriarchy has lied to men about what is and is not OK, in sex and in gender relations as a whole. Men are now being caught out and unfairly punished for their innocent mistakes, as women enforce a new set of rules. Perhaps these new rules are the correct ones; and, doubtless, the old ones caused a lot of harm. But how were men to have known any better? In their minds they weren’t guilty, so don’t they too have grounds for acquittal? 21
\end{quote}
\vskip 2ex

the example of the “Shitty Media Men” list (2017) 22ff.
\vskip 2ex

What does it really take to alter the mind of patriarchy?
\begin{quote}
	And yet, if the aim is not merely to punish male sexual domination but to end it, feminism must address questions that many feminists would rather avoid: whether a carceral approach that systemically harms poor people and people of colour can serve sexual justice; whether the notion of due process -- and perhaps too the presumption of innocence -- should apply to social media and public accusations; whether punishment produces social change.\\[1ex] 
	What does it really take to alter the mind of patriarchy? 24
\end{quote}
\vskip 2ex

Kwadwo Bonsu, University of Massachusetts, 2014 case (24ff.) -- an expectation already internalised
\begin{quote}
	She kept going for the reason that so many girls and women keep going: because women who sexually excite men are supposed to finish the job. It doesn’t matter whether Bonsu himself had this expectation, because it is \textbf{an expectation already internalised by many women}. A woman going on with a sex act she no longer wants to perform, knowing she can get up and walk away but knowing at the same time that this will make her \textbf{a blue-balling tease}, an object of male contempt: \textbf{there is more going on here than mere ambivalence, unpleasantness and regret}. 27f.
\end{quote}
\vskip 2ex

a kind of coercion, by the informal regulatory system of gendered sexual expectations
\begin{quote}
	There is also \textbf{a kind of coercion}: not directly by Bonsu, perhaps, but \textbf{by the informal regulatory system of gendered sexual expectations}. Sometimes the price for violating these expectations is steep, even fatal.\\[1ex] 
	That is why there is a connection between these episodes of ‘ordinary’ sex and the ‘actual wrongs and harms’ of sexual assault. What happened at UMass may well be ‘ordinary’ in the statistical sense -- as in what happens every day -- but it isn’t ‘ordinary’ in the ethical sense, as in what we should pass over without comment. In that sense it is \textbf{an extraordinary phenomenon with which we are all too familiar}. 28
\end{quote}
\vskip 2ex

California SB 967, the ‘Yes Means Yes’ bill, 2014: ‘affirmative consent’ --\\
How to formulate a regulation that prohibits the sort of sex that is produced by patriarchy?
\begin{quote}
	As Catharine MacKinnon has pointed out, affirmative consent laws simply shift the goalposts on what constitutes legally acceptable sex: whereas previously men had to stop when women said no, now they just have to get women to say yes.\\[1ex] 
	\textbf{How do we formulate a regulation that prohibits the sort of sex that is produced by patriarchy?}\\[1ex] 
	Could the reason that this question is so hard to answer be that the law is simply the wrong tool for the job? 29
\end{quote}
\vskip 2ex

a feminism worth having -- avoid re-enactment of crime and punishment
\begin{quote}
	I am not saying that feminism has no business asking better of men -- indeed, asking them to be better men. But \textbf{a feminism worth having must find ways of doing so that avoid rote re-enactment of the old form of crime and punishment, with its fleeting satisfactions and predictable costs}.\\[1ex] 
	I am saying that a feminism worth having must, not for the first time, expect women to be better -- not just fairer, but more imaginative -- than men have been. 30
\end{quote}
\vskip 2ex

\begin{quote}
	the genre that Jia Tolentino calls ‘My Year of Being Held Responsible for My Own Behavior’ 31
\end{quote}
\vskip 5ex
\pagebreak

\begin{center}
	{\large Talking to My Students About Porn}
\end{center}
\vskip 3ex

\begin{quote}
	\textbf{Did porn kill feminism?} \\[1ex] 
	That’s one way of telling the story of the US women’s liberation movement [\dots\hspace{-0.3ex}]\\[1ex]
	Debates about \textbf{porn} -- 
	\begin{itemize}
	\item is it \textbf{a tool of patriarchy}, \protect[\textbf{a technique of subordination},\protect] or
	\item \protect[\textbf{a counter to sexual repression},\protect] \textbf{an exercise of free speech}?
	\end{itemize}
	 -- came to preoccupy the women’s liberation movement in the US, and to some degree the UK and Australia, and then to tear it apart. 33
\end{quote}
\vskip 2ex

\textbf{Barnard Sex Conference}, NYC, April 24, 1982 (presented as the annual Scholar and Feminist Conference IX):\\[1ex]
Carole S. Vance, “CONCEPT PAPER: Towards a Politics of Sexuality” (January, 1982)
\begin{quote}
	In the conference’s concept paper, ‘Towards a Politics of Sexuality’, Carole Vance called for an acknowledgement of sex as ‘simultaneously a domain of restriction, repression, and danger as well as a domain of exploration, pleasure, and agency’. 33
\end{quote}
\vskip 2ex

the ‘porn question’, settled
\begin{quote}
	the internet has settled the ‘porn question’ for us 35
\end{quote}
\vskip 2ex

‘porn’, as in: ‘problematic’ sex in general
\begin{quote}
	porn came to serve, for feminists of an earlier generation, as a metonym for ‘problematic’ sex in general: for sex that took no account of women’s pleasure, for sadomasochistic sex, for prostitution, for rape fantasies, for sex without love, for sex across power differentials, for sex with men. 35
\end{quote}
\vskip 2ex

Catherine MacKinnon, \emph{Only Words} (1993) --- porn as a mechanism not just for depicting the world, but for making it; porn’s world-making power
\begin{quote}
	To say that it is porn’s function to \emph{effectuate} its message is to see \textbf{porn as a mechanism not just for depicting the world, but for making it}.\\[1ex] Porn, for MacKinnon and other anti-porn feminists, was a machine for the production and reproduction of an ideology which, by eroticising women’s subordination, thereby made it real. 38
\end{quote}
\vskip 2ex

black feminists -- mainstream porn’s canonical female persona: ‘the demure slut’
\begin{quote}
	They identified the template for m stream pornography in the historical display of black women’s bodies in the contexts of colonialism and slavery [\dots\hspace{-0.3ex}]\\[1ex]
	In her classic \emph{Black Feminist Thought} (1990), Patricia Hill Collins identified a precursor to the white female pornographic object in \textbf{the mixed-race slave women who were specifically ‘bred’ to be indistinguishable from white women}.\\[1ex]
	These women \textbf{‘approximated the images of beauty, asexuality, and chastity forced on white women,’} Collins wrote, \textbf{but ‘inside was a highly sexual whore, a “slave mistress” ready to cater to her owner’s pleasure.’} It is from this racialised and gendered practice, Collins suggested, that m\textbf{ainstream porn got its canonical female persona: the demure slut}. 38f.
\end{quote}
\vskip 2ex

\begin{quote}
	But what if the true significance of the perspective of anti-porn feminists lay not in what they were paying attention to, but when?\\[1ex] 
	What if they weren’t hysterical, but prescient? 39f.
\end{quote}
\vskip 2ex

my students
\begin{quote}
	Could it be that p raphy doesn’t merely depict the subordination of women, but actually makes it real, I asked? Yes, they said.\\[1ex] 
	Does porn silence women, making it harder for them to protest against unwanted sex, and harder for men to hear those protests? Yes, they said.\\[1ex] 
	Does porn bear responsibility for the objectification of women, for the marginalisation of women, for sexual violence against women?Yes, they said, yes to all of it. 40
\end{quote}
\vskip 2ex

\begin{quote}
	‘But if it weren’t for pornography,’ one woman said, ‘how would we ever learn to have sex?’ 40
\end{quote}
\vskip 2ex

a script in place: sex for my students is what porn says it is
\begin{quote}
	My students would not have stolen or passed around magazines or videos, or gathered glimpses here and there. For them sex was there, fully formed, fully interpreted, \textbf{fully categorised} -- \emph{teen, gangbang, MILF, stepdaughter} -- \textbf{waiting on the screen}.\\[1ex] 
	By the time my students got around to sex IRL -- later, it should be noted, than teenagers of previous generations -- there was, at least for the straight boys and girls, \textbf{a script in place} that dictated not only the physical moves and gestures and sounds to make and demand, but also \textbf{the appropriate affect, the appropriate desires, the appropriate distribution of power}.\\[1ex]
	\textbf{The psyches of my students are products of pornography}.\\[1ex] 
	In them, the warnings of the anti-porn feminists seem to have been belatedly realised: \textbf{sex for my students is what porn says it is}. 41
\end{quote}
\vskip 2ex

Catherine MacKinnon, \emph{Only Words}:
\begin{quote}
	the consumers want to live out the pornography further in three dimensions [\dots\hspace{-0.3ex}]\\[1ex]
	--- using and making pornography is inextricable to these acts 42
\end{quote}
\vskip 2ex

the porn watcher
\begin{quote}
	a startling image: porn as a virtual training ground for male sexual aggression. Could it be true?\\[1ex] 
	\textbf{Or is this image itself a kind of sexual fantasy, which reduces misogyny to a single origin, and its many, diverse agents to a single subject: the porn watcher?} 42
\end{quote}
\vskip 2ex

\begin{quote}
	What will the world look like once another generation or two has passed, when every person on Earth will have come of age sexually in the pornworld? 44
\end{quote}
\vskip 2ex

porn as the normative standard of sex
\begin{quote}
	\emph{You’re doing it wrong.} Porn was for this young man \textbf{the normative standard of sex}, against which his girlfriend was measured and found wanting.\\[1ex] 
	Porn is not pedagogy, yet it \textbf{often functions as if it were}. 44
\end{quote}
\vskip 2ex

Is it porn’s responsibility to tell the truth?
\begin{quote}
	Porn may tell lies about sex and women -- in John Stoltenberg’s famous formula, \textbf{‘pornography tells lies about women’ but ‘tells the truth about men’} -- but so what?\\[1ex] 
	\textbf{Is it porn’s responsibility to tell people}, especially young people, \textbf{the truth about sex?} 45
\end{quote}
\vskip 2ex

anti-porn feminism: porn itself an act of subordinating women
\begin{quote}
	Crucial to anti-porn feminism is the thought that porn doesn’t just happen to result in women’s subordination: \textbf{it is itself an act of subordinating women}. Specifically, pornography performs the speech act of licensing the subordination of women, and conferring on women an inferior civic status. Like the stampede that follows my shouting ‘Fire!’, \textbf{porn’s effects on women are not just}, anti-porn feminists think, \textbf{the expected result, but moreover the whole point, of pornography}. 46
\end{quote}
\vskip 2ex

the appeal to childhood innocence
\begin{quote}
	The \textbf{appeal to childhood innocence} also tends to draw an implausibly sharp distinction between the way things were and the way things are now, skating over the continuities: between the Rolling Stones and Miley Cyrus, between top-shelf magazines and PornHub, between making out in the back row and the dick pic.\\[2ex]
	What’s more, it is arguably the rest of us, and \textbf{not today’s teenagers and young adults, who are under-equipped to deal with the technological renovation of our social world}. By this I don’t just mean that kids are the ones who most easily \textbf{grasp the semiotic possibilities of TikTok and Instagram}.\\[1ex] 
	I also mean that they have \textbf{a sensitivity to the workings of gendered and racialised power that outstrips anything seen before in the political mainstream}. It would be a mistake to assume that they are unable to cope with the pornworld just because we believe that we, as children, couldn’t have coped.\\[1ex] 
	Like the anti-porn feminists of the second wave, perhaps my students attribute too much power to porn, and have too little faith in their ability to resist it. 47f.
\end{quote}
\vskip 2ex

Peggy Orenstein, \emph{Girls \& Sex. Navigating the Complicated New Landscape} (HarperCollins 2017).
\begin{quote}
	They [the young women Orenstein discusses in \emph{Girls \& Sex}] would not have been ashamed, as I and all my friends were, to call themselves feminists. \textbf{How should we understand the relation between this raised state of feminist consciousness among young women, and what appear to be their worsening sexual conditions: increased objectification, intensified body expectations, decreasing pleasure, and shrinking options for sex on their terms?}\\[1ex]
	Perhaps girls and young women are becoming more feminist because their worsening circumstances demand it.\\[1ex] 
	Or perhaps, as Orenstein suggests, feminist consciousness is for many young women a mode of false consciousness, which plays into the hands of the very system of sexual subordination they take themselves to be opposing.\\[1ex] 
	Does a discourse of sexual empowerment and autonomy mask something darker and unfree? 49
\end{quote}
\vskip 2ex

\begin{quote}
	The situation Orenstein described was, they [seventeen-year-old girls at a London school] said, their own: a life of sex without dating, where girls gave and boys received, and where a discourse of empowerment and body confidence masked a deeper sense of disappointment and shame. 50
\end{quote}
\vskip 2ex

pornography as speech
\begin{quote}
	To say that pornography is speech is, in a liberal jurisdiction such as the US, to say that porn is deserving of special protection.\\[1ex] 
	Freedom of speech is connected to many things liberal societies value (or claim to): individual autonomy, the democratic accountability of the government, the sanctity of personal conscience, tolerance of difference and disagreement, the pursuit of truth.\\[1ex]In the US speech is given unusually strong protection, and the notion itself ‘speech’ -- is interpreted with unusual breadth. 52
\end{quote}
\vskip 2ex

free speech, restricted only by form, not by content
\begin{quote}
	While this viewpoint [sc. burning crosses] might be a rent, [sc. Supreme Court justice] Scalia [sc. re St Paul Bias-Motivated Crime Ordinance, 1992] reasoned, it was \textbf{still a viewpoint}, whose expression must therefore be protected. The only permissible restrictions on speech were grounded, Scalia insisted, on the \emph{form} that speech took -- for example, knowingly false speech (libel, defamation), or speech that involved the criminal abuse of children for its production (child pornography).\\[1ex] 
	Racist or sexist speech could not be prohibited or suppressed on the grounds of its content, for then the state would be intervening in \textbf{the free marketplace of ideas}. [\dots\hspace{-0.3ex}]\\[1ex]
	in the ‘debate’ between white supremacists and black people over racial equality, the state couldn’t take sides 53
\end{quote}
\vskip 2ex

the mainstream pornographers’ right to express their viewpoint
\begin{quote}
	A similar argument was mobilised by judges and legal scholars against the Dworkin-MacKinnon anti-porn ordinances. The legislation, they argued, violated the right of mainstream pornographers to express their viewpoint that women were objects for the sexual use of men.\\[1ex] 
	Since the Dworkin-MacKinnon ordinances did not target all pornographic material, but only pornographic material that subordinated women by presenting them as dehumanised sexual objects, it discriminated on the basis of content rather than form. In the debate between misogynists and feminists over women’s equality, the state couldn’t take sides.
\end{quote}
\vskip 2ex

MacKinnon’s rebuttal
\begin{quote}
	First, porn’s ‘contribution’ to the debate about women’s status precludes the possibility of women’s entering the debate on equal terms. [\dots\hspace{-0.3ex}]\\[1ex]
	The exercise of pornographers’ right to free speech undermines women’s own right to free speech. 53\\[1ex]
	Second, MacKinnon argued, pornography doesn’t merely express the view that women are to be subordinated -- it is not ‘only words’. By training our attention on porn and its worldly effects, we can come to see it as \textbf{an act of subordination, whose function is to enforce the second-class status of all women in relation to men}. The very fact that judges, lawyers and philosophers insist on \textbf{treating porn as a question of free speech -- as a question of what porn says rather than what it does --} betrays their implicitly male perspective, their failure to see porn as many women see it. 54
\end{quote}
\vskip 2ex

‘free speech’ as an ideological tool --- the distinction speech v/ action
\begin{quote}
	‘Free speech’, which poses as a merely formal principle of adjudication, is in fact, MacKinnon suggests, an ideological tool selectively deployed to protect the freedoms of the dominant class.\\[1ex] 
	(This is something that the feminist philosophers who have sought to elaborate and defend MacKinnon’s argument generally miss: \textbf{the issue, for MacKinnon, is not that pornography really is, metaphysically speaking, an action rather than mere speech, but that the very distinction between speech and action is political all the way down}.) 54
\end{quote}
\vskip 2ex

\emph{R. v. Butler}, Ontario, 1992. 54f.
\vskip 2ex

feminism x the state x the New Right
\begin{quote}
	The early feminist campaigns against porn in the 1960s took direct action against the makers and sellers of porn in the form of boycotts and protests. By contrast, the antiporn campaigners of the early 1980s called on the power of the state. [\dots\hspace{-0.3ex}]\\[1ex]
	Should it have come as a surprise when the state, under the cover of feminism, acted to further the subordination of women and sexual minorities?\par
	This question had a particular significance in the late 1970s and early 1980s, when the US anti-porn feminists were campaigning.\\[1ex] 
	The decision of the US Supreme Court in \emph{Roe v. Wade} (1973) to legalise abortion represented a significant victory for feminism,\\[1ex] 
	but also led to an organised right-wing backlash which united, to determining and lasting effect, religious conservatives with proponents of neoliberal economics.\\[1ex] 
	Central to the New Right’s ideological programme was a reversal of feminist achievements: not just the legalisation of abortion, but also the availability of contraception and birth control, sex education, gay and lesbian rights, and women’s mass entry into the workforce.\\[1ex] 
	In this climate, radical feminist critiques of pornography dovetailed with a conservative ideology which made a distinction between ‘bad’ women (sex workers, ‘welfare queens’) who must be disciplined by the state and ‘good’ women who needed its protection, and which saw men as naturally rapacious and in need of taming by the institutions of monogamous marriage and the nuclear family. [\dots\hspace{-0.3ex}]\\[1ex]
	It was Ronald Reagan, the lodestar of the New Right, who as president ordered his attorney general to conduct an investigation into the harms of pornography, to which MacKinnon and Dworkin gave expert testimony. 55f.
\end{quote}
\vskip 2ex

UK 2014 legislation, ban on non-normative pornographic acts
\begin{quote}
	what is officially sanctioned here, by virtue of being left off the list, is the most mainstream porn, the porn that turns most people on.\\[1ex] 
	But the whole point of the feminist critiques of porn was \textbf{to disrupt the logic of the mainstream: to suggest that what turns most people on is not thereby OK.} To prohibit only what is marginal in sex is to reinforce the hegemony of mainstream sexuality: to reinforce mainstream misogyny. 58
\end{quote}
\vskip 2ex

porn is going to be made
\begin{quote}
	Whatever the law says, porn is going to be made, bought and sold. [?!]\\[1ex] 
	What should matter most to feminists is not what the law says about porn, but what the law does for and to the women who work in it. 62
\end{quote}
\vskip 2ex

the internet cannot be contained
\begin{quote}
	Not one of my students, in the now several years I have been teaching seminars on porn, has suggested using legislation to mitigate its effects. This isn’t because my students are free speech fanatics. It’s because \textbf{they are pragmatists}. \textbf{They instinctively know that the internet cannot be contained}, and that blocking access to it may work on members of older, less savvy generations, but not on theirs. 62
\end{quote}
\vskip 2ex

legislate or educate?
\begin{quote}
	In their [my students’] view, porn has the power to teach them the truth about sex not because the state has failed to legislate, but because the state has failed in its basic responsibility to educate. 62
\end{quote}
\vskip 2ex

‘porn literacy’ (UK) 62
\vskip 2ex

the appeal to the law, the appeal to education 63
\vskip 2ex

porn trains
\begin{quote}
	sex education, traditionally conceived [sc. asking the educated to deliberate, question and understand], does not propose to meet porn on its own ground.\\[1ex] 
	\textbf{For porn does not inform, or persuade, or debate. Porn trains.} It etches deep grooves in the psyche, forming powerful associations between arousal and selected stimuli, bypassing that part of us which pauses, considers, thinks. 64
\end{quote}
\vskip 2ex

the pornworld
\begin{quote}
	In front of the porn film, the imagination halts and gives way, overtaken by its simulacrum of reality. The browser window is transformed into a window onto the world, \textbf{the pornworld}, in which slick bodies fuck and are fucked for their own pleasure. 64
\end{quote}
\vskip 2ex

the porn film \emph{as} film: the pleasures of looking and listening
\begin{quote}
	The pleasures afforded by the porn film \emph{as} a film are those afforded by any other: \textbf{the pleasures of looking and listening}. 64
\end{quote}
\vskip 2ex


mainstream porn, the pleasures of ego-identification
\begin{quote}
	Except that, in reality, there is no pornworld and no window onto it, and there is nothing incidental about the pleasures we take from porn. \textbf{Porn is an elaborate construction designed to get the viewer off}. That the sex in it might be real, and that the pleasure sometimes is too, doesn’t change this.\\[1ex] 
	Obviously, mainstream porn offers \textbf{the pleasures of looking at the woman’s body on display}, its orifices, one by one, awaiting penetration: mouth, vagina, anus.\\[1ex] 
	But, more than this, it offers \textbf{the pleasures of ego-identification}. For mainstream porn depicts a very particular kind of sexual schema in which, on the whole, women are hungry for the assertion of male sexual power -- and then assigns to the viewer a particular focus of identification within it. \textbf{64f.}
\end{quote}
\vskip 2ex













\end{document}