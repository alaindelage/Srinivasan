% \documentclass[a4paper]{report}
\input{preamble}
\input{hyphenation}

%\setlength{\baselineskip}{2ex}
% \renewcommand{\baselinestretch}{1.2}
\selectlanguage{ngerman}
\usepackage{fourier}

\begin{document}
% \doublespacing
% \onehalfspacing
\thispagestyle{empty}

% einzeiliger Titel

% {\Large AUTOR\ \ \emph{TITEL} (JAHR)}

% oder mehrzeiliger Titel

\begin{tabbing}
	{\Large Amia Srinivasan} \ \ \={\Large \emph{The Right to Sex}}\\[1ex]
	\> {\large \emph{Feminism in the Twenty-First Century} (Bloomsbury 2021)}
\end{tabbing}
\vskip 5ex


% \flushleft
\RaggedRight

% \changefont{ptm}{m}{n}
\begin{center}
	{\large Preface}
\end{center}
\vskip 3ex

feminism: a political movement, not a theory
\begin{quote}
	\textbf{Feminism} is \textbf{not a philosophy}, or \textbf{a theory}, or even \textbf{a point of view}.\\[1ex] 
	It is \textbf{a political movement to transform the world beyond recognition}. It asks: what would it be to end the political, social, sexual, economic, psychological and physical subordination of women? \textbf{It answers: we do not know; let us try and see}. xi
\end{quote}
\vskip 2ex

‘sex’
\begin{quote}
	Feminism begins with \textbf{a woman’s recognition that she is a member of a sex class}: that is, a member of a class of people \textbf{assigned to an inferior social status on the basis of something called ‘sex’} -- \textbf{a thing that is said to be natural, pre-political, an objective material ground on which the world of human culture is built}. xi
\end{quote}
\vskip 2ex

‘sex’, this supposedly natural thing -- a cultural thing posing as a natural one
\begin{quote}
	We inspect \textbf{this supposedly natural thing, ‘sex’}, only to find that it is \textbf{already laden with meaning}.\\[1ex] 
	At birth, bodies are sorted as ‘male’ or ‘female’, though many bodies must be mutilated to fit one category or the other, and many bodies will later protest against the decision that was made.\\[1ex] 
	\textbf{This originary division determines what social purpose a body will be assigned}. [\dots\hspace{-0.3ex}]\\[1ex]
	\textbf{Sex} is, then, \textbf{a cultural thing posing as a natural one}. Sex, which feminists have taught us to distinguish from gender, is \textbf{itself already gender in disguise}. xi f.
\end{quote}
\vskip 2ex

‘sex’, in another sense: a thing we do with our sexed bodies
\begin{quote}
	\textbf{‘sex’}: sex as \textbf{a thing we do with our sexed bodies}. Some bodies are for other bodies to have sex with. [\dots\hspace{-0.3ex}]\\[1ex] 
	‘Sex’ in this second sense is also \textbf{said to be a natural thing, a thing that exists outside politics}. Feminism shows that \textbf{this too is a fiction}, and a fiction that serves certain interests. Sex, which we think of as the most private of acts, is \textbf{in reality a public thing}. [\dots\hspace{-0.3ex}]\\[1ex]
	\textbf{the rules for all this were set long before we entered the world} xii
\end{quote}
\vskip 2ex

feminism ans sexual freedom
\begin{quote}
	Feminists have long dreamed of \textbf{sexual freedom}. What they refuse to accept is \textbf{its simulacrum: sex that is said to be free, not because it is equal, but because it is ubiquitous}. In this world, sexual freedom is \textbf{not a given but something to be achieved}, and it is \textbf{always incomplete}. xii
\end{quote}
\vskip 2ex

\begin{quote}
	\textbf{What would it take for sex really to be free?} We do not yet know; let us try and see. xiii
\end{quote}
\vskip 3ex
\pagebreak

sex as a political phenomenon -- beyond the narrow parameters of ‘consent’
\begin{quote}
	These essays are \textbf{about the politics and ethics of sex in this world}, animated by a hope of a different world.\\[1ex] 
	They reach back to \textbf{an older feminist tradition that was unafraid to think of sex as a political phenomenon}, as something \textbf{squarely within the bounds of social critique}. The women in this tradition -- from Simone de Beauvoir and Alexandra Kollontai to bell hooks, Audre Lorde, Catharine MacKinnon and Adrienne Rich – dare us to think about the ethics of sex \textbf{beyond the narrow parameters of ‘consent’}. They compel us to ask \textbf{what forces lie behind a woman’s \emph{yes}}; what it reveals about sex that it is something to which consent must be given; how it is that we have come to put so much psychic, cultural and legal weight on a notion of ‘consent’ that cannot support it.\\[1ex] 
	And they ask us to join them in \textbf{dreaming of a freer sex}. xiii
\end{quote}
\vskip 2ex

remake the political critique of sex for the twenty-first century
\begin{quote}
	At the same time, these essays seek to \textbf{remake the political critique of sex for the twenty-first century}: to take seriously \textbf{the complex relationship of sex to race, class, disability, nationality and caste}; to think about what sex has become \textbf{in the age of the internet}; to ask what it means to \textbf{invoke the power of the capitalist and carceral state} to address the problems of sex. xiii f.
\end{quote}
\vskip 2ex

feminism not as ‘home’
\begin{quote}
	Feminism cannot indulge the fantasy that interests always converge; [\dots\hspace{-0.3ex}]\\[1ex]
	\textbf{Feminism envisaged as a ‘home’ insists on commonality before the fact} [\dots\hspace{-0.3ex}] A truly inclusionary politics is an uncomfortable, unsafe politics. xv
\end{quote}
\vskip 2ex

\begin{quote}
	they [the essays] represent my attempt to put into words what many women, and some men, already know xv
\end{quote}
\vskip 5ex
\pagebreak

\begin{center}
	{\large The Conspiracy Against Men}
\end{center}
\vskip 3ex

false rape accusation -- and its cultural charge
 \begin{quote}
 	Nonetheless, a false rape accusation, like a plane crash, is an objectively unusual event that occupies an outsized place in the public imagination.\\[1ex]
 	\textbf{Why then does it carry its cultural charge?} 3
 \end{quote}
 \vskip 2ex
 
 false rape accusation, by men
\begin{quote}
	very often, it is men who falsely accuse other men of raping women. This is a thing almost universally misunderstood about false rape accusations. When we think of a false rape accusation we picture a scorned or greedy woman, lying to the authorities. But many, perhaps most, wrongful convictions of rape result from false accusations levied against men by other men 4
\end{quote}
\vskip 2ex

false rape accusations as a predominantly wealthy white male preoccupation
\begin{quote}
	It might seem surprising, then, that false rape accusations are, today, a predominantly wealthy white male preoccupation.\\[1ex]
	But it isn’t surprising -- not really. The \textbf{anxiety about false rape accusations} is purportedly about injustice (innocent people being harmed), but \textbf{actually it is about gender, about innocent men being harmed by malignant women}. It is \textbf{an anxiety}, too, \textbf{about race and class}: about \textbf{the possibility that the law might treat wealthy white men as it routinely treats poor black and brown men}.\\[1ex] 
	For poor men, and women, of colour, the white woman’s false rape accusation is just one element in a matrix of vulnerability to state power.\\[1ex]
	false rape accusations are \textbf{a unique instance of middle-class and wealthy white men’s vulnerability} to the injustices routinely perpetrated by the carceral state against poor people of colour. 5f.
\end{quote}
\vskip 2ex

the representation is false -- but, as ideologically efficacious
\begin{quote}
	That representation is, of course, false: even in the case of rape, the state is on the side of wealthy white men.\\[1ex] 
	But what matters in the sense of what is ideologically efficacious -- is not the reality, but the misrepresentation. \textbf{In the false rape accusation, wealthy white men misperceive their vulnerability to women and to the state}. 6
\end{quote}
\vskip 2ex

Brock Turner case (Santa Clara County, 2016)
\begin{quote}
	‘20 minutes of action’ -- healthy, adolescent fun [\dots\hspace{-0.3ex}]\\[1ex]
	in a sense Dan Turner is talking about an animal, a perfectly bred specimen of wealthy white American boyhood [\dots\hspace{-0.3ex}]\\[1ex]
	like an animal, Brock is imagined to exist outside the moral order. These red-blooded, white-skinned, all-American boys [\dots\hspace{-0.3ex}] are good kids, the best kids, our kids. 7
\end{quote}
\vskip 2ex

Brett Kavanaugh
\begin{quote}
	The solidarity on show from the people who knew Kavanaugh when young – what Kavanaugh calls ‘friendship’ -- was the solidarity of rich white people.\\[1ex] 
	We can’t imagine a black or brown Kavanaugh without inverting America’s racial and economic rules. 9
\end{quote}
\vskip 2ex

‘Believe women’, \#IBelieveHer
\begin{quote}
	Whom are we to believe, the white woman who says she was raped, or the black or brown woman who insists that her son is being set up? Carolyn Bryant or Mamie Till? 9
\end{quote}
\vskip 2ex

dismissal of ‘Believe women’ as a category error
\begin{quote}
	a political response to what we suspect will be its [legal principle of the presumption of innocence] uneven application [\dots\hspace{-0.3ex}]\\[1ex]
	Against this prejudicial enforcement of the presumption of innocence, ‘Believe women’ operates as a corrective norm, a gesture of support for those people – women – whom the law tends to treat as if they were lying. 9
\end{quote}
\vskip 2ex

a category error in a second sense
\begin{quote}
	The law must address each individual on a case-by-case basis [\dots\hspace{-0.3ex}] but the norms of the law do not set the norms of rational belief. Rational belief is proportionate to the evidence [\dots\hspace{-0.3ex}]\\[1ex]
	the outcome of a trial does not determine what we should believe 10
\end{quote}
\vskip 2ex

why sex crimes elicit such selective scepticism?
\begin{quote}
	The question, from a feminist perspective, is \textbf{why sex crimes elicit such selective scepticism}.\\[1ex] 
	And the answer that feminists should give is that \textbf{the vast majority of sex crimes are perpetrated by men against women}.\\[1ex] 
	Sometimes, the injunction to ‘Believe women’ is simply the injunction to form our beliefs in the ordinary way: in accordance with the facts. 10f.
\end{quote}
\vskip 2ex

\begin{quote}
	Does ‘Believe women’ serve justice at Colgate [University, elite liberal arts college; 4.2 per cent of the student body black in the academic year 2013-14; yet 50 per cent of accusations of sexual violation against black students]? 11
\end{quote}
\vskip 2ex

Jyoti Singh, 16 December 2012, Delhi
\begin{quote}
	the brutality of the attack on Jyoti Singh was cited by non-Indians as a way of disavowing any commonality between the sexual cultures of India and their own countries. [\dots\hspace{-0.3ex}]\\[1ex]
	A first question: why is it that when white men rape they are violating a norm, but when brown men rape they are conforming to one?\\[1ex]
	A second question: if Indian men are hyenas, what does that make Indian women? 12
\end{quote}
\vskip 2ex

the spectacle of the black male corpse, v/ the lack of the spectacle of the black female corpse
\begin{quote}
	‘What,’ Threadcraft asks, ‘will motivate people to rally around the bodies of our black female dead?’ 14
\end{quote}
\vskip 2ex

the white mythology about black sexuality
\begin{quote}
	portraying black men as rapists and black women as unrapeable 14
\end{quote}
\vskip 2ex

the doubled sexual subordination of black women
\begin{itemize}
	\item Black women who speak out against black male violence are blamed for reinforcing negative stereotypes of their community and for calling on a racist state to protect them. 

	\item At the same time, the internalisation of the sexually precocious black girl stereotype means that black girls and women are seen by some black men as asking for their abuse. 14
\end{itemize}
\vskip 2ex

Clarence Thomas v/ Anita Hill
\begin{quote}
	Fairfax did not note the irony of comparing black women to a white lynch mob.\\[1ex] Neither did Clarence Thomas, for that matter, when he accused Anita Hill in 1991 of triggering a ‘high-tech lynching’.\\[1ex] 
	\textbf{The very logic that made the lynching of black men possible -- the logic of black hypersexuality -- is repurposed, at the level of metaphor, to falsely indict black women as the true oppressors}. 15
\end{quote}
\vskip 2ex

intersectionality’s central insight: the danger of a lib movement’s serving the least suppressed members of a group
\begin{quote}
	\textbf{‘Intersectionality’} -- a term coined by \textbf{Kimberlé Crenshaw} to name an idea first articulated by an older generation of feminists from Claudia Jones to Frances M. Beal, the Combahee River Collective, Selma James, Angela Davis, bell hooks, Enriqueta Longeaux y Vásquez and Cherríe Moraga -- is \textbf{often reduced}, in common understanding,\textbf{ to a due consideration of the various axes of oppression and privilege: race, class, sexuality, disability and so on}. But to reduce intersectionality to a mere attention to difference is to forego its power as a theoretical and practical orientation.\\[1ex]
	\textbf{The central insight of intersectionality is that any liberation movement -- feminism, anti-racism, the labour movement -- that focuses only on what all members of the relevant group (women, people of colour, the working class) have in common is a movement that will best serve those members of the group who are least oppressed}. 17
\end{quote}
\vskip 2ex

the dilemma of ‘Believe women’
\begin{quote}
	When we are too quick to believe a white woman’s accusation against a black man, or a Brahmin woman’s accusation against a Dalit man, it is black and Dalit women who are rendered more vulnerable to sexual violence. Their ability to speak out against the violence they face from men of their race or caste is stifled, and their status as counterpart to the oversexed black or Dalit male is entrenched. In that paradox of female sexuality, such women are rendered unrapeable, and thus more rapeable. 18
\end{quote}
\vskip 2ex

\#Metoo, or, the rules changed suddenly
\begin{quote}
	This idea -- that the rules have suddenly changed on men, so that they now face punishment for behaviour that was once routinely permitted -- has become a \#MeToo commonplace. 20\\[1ex]
	\#MeToo is often seen as having produced a generalised version of the situation in which John Cogan found himself. Patriarchy has lied to men about what is and is not OK, in sex and in gender relations as a whole. Men are now being caught out and unfairly punished for their innocent mistakes, as women enforce a new set of rules. Perhaps these new rules are the correct ones; and, doubtless, the old ones caused a lot of harm. But how were men to have known any better? In their minds they weren’t guilty, so don’t they too have grounds for acquittal? 21
\end{quote}
\vskip 2ex

the example of the “Shitty Media Men” list (2017) 22ff.
\vskip 2ex
\pagebreak

What does it really take to alter the mind of patriarchy?
\begin{quote}
	And yet, if the aim is not merely to punish male sexual domination but to end it, feminism must address questions that many feminists would rather avoid: whether a carceral approach that systemically harms poor people and people of colour can serve sexual justice; whether the notion of due process -- and perhaps too the presumption of innocence -- should apply to social media and public accusations; whether punishment produces social change.\\[1ex] 
	\textbf{What does it really take to alter the mind of patriarchy?} 24
\end{quote}
\vskip 2ex

Kwadwo Bonsu, University of Massachusetts, 2014 case (24ff.) -- an expectation already internalised
\begin{quote}
	She kept going for the reason that so many girls and women keep going: because women who sexually excite men are supposed to finish the job. It doesn’t matter whether Bonsu himself had this expectation, because it is \textbf{an expectation already internalised by many women}. A woman going on with a sex act she no longer wants to perform, knowing she can get up and walk away but knowing at the same time that this will make her \textbf{a blue-balling tease}, an object of male contempt: \textbf{there is more going on here than mere ambivalence, unpleasantness and regret}. 27f.
\end{quote}
\vskip 2ex

a kind of coercion, by the informal regulatory system of gendered sexual expectations
\begin{quote}
	There is also \textbf{a kind of coercion}: not directly by Bonsu, perhaps, but \textbf{by the informal regulatory system of gendered sexual expectations}. Sometimes the price for violating these expectations is steep, even fatal.\\[1ex] 
	That is why there is a connection between these episodes of ‘ordinary’ sex and the ‘actual wrongs and harms’ of sexual assault. What happened at UMass may well be ‘ordinary’ in the statistical sense -- as in what happens every day -- but it isn’t ‘ordinary’ in the ethical sense, as in what we should pass over without comment. In that sense it is \textbf{an extraordinary phenomenon with which we are all too familiar}. 28
\end{quote}
\vskip 2ex

California SB 967, the ‘Yes Means Yes’ bill, 2014: ‘affirmative consent’ --\\
How to formulate a regulation that prohibits the sort of sex that is produced by patriarchy?
\begin{quote}
	As Catharine MacKinnon has pointed out, affirmative consent laws simply shift the goalposts on what constitutes legally acceptable sex: whereas previously men had to stop when women said no, now they just have to get women to say yes.\\[1ex] 
	\textbf{How do we formulate a regulation that prohibits the sort of sex that is produced by patriarchy?}\\[1ex] 
	Could the reason that this question is so hard to answer be that the law is simply the wrong tool for the job? 29
\end{quote}
\vskip 2ex

a feminism worth having -- avoid re-enactment of crime and punishment
\begin{quote}
	I am not saying that feminism has no business asking better of men -- indeed, asking them to be better men. But \textbf{a feminism worth having must find ways of doing so that avoid rote re-enactment of the old form of crime and punishment, with its fleeting satisfactions and predictable costs}.\\[1ex] 
	I am saying that a feminism worth having must, not for the first time, expect women to be better -- not just fairer, but more imaginative -- than men have been. 30
\end{quote}
\vskip 2ex

\begin{quote}
	the genre that Jia Tolentino calls ‘My Year of Being Held Responsible for My Own Behavior’ 31
\end{quote}
\vskip 2ex





















\end{document}